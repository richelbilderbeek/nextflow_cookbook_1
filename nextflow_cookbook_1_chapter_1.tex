%%%%%%%%%%%%%%%%%%%%%%%%%%%%%%%%%%%%%%%%%%%%%%%%%%%%%%%%%%%%%%%%%%%%%%%%%%%%%%%%
\chapter{Introduction}
%%%%%%%%%%%%%%%%%%%%%%%%%%%%%%%%%%%%%%%%%%%%%%%%%%%%%%%%%%%%%%%%%%%%%%%%%%%%%%%%

This is 'Nextflow Cookbook 1: Basics', version 0.1.

%%%%%%%%%%%%%%%%%%%%%%%%%%%%%%%%%%%%%%%%%%%%%%%%%%%%%%%%%%%%%%%%%%%%%%%%%%%%%%%%
\section{Why this tutorial}
%%%%%%%%%%%%%%%%%%%%%%%%%%%%%%%%%%%%%%%%%%%%%%%%%%%%%%%%%%%%%%%%%%%%%%%%%%%%%%%%

This cookbook started out as a collection of notes to myself.

%%%%%%%%%%%%%%%%%%%%%%%%%%%%%%%%%%%%%%%%%%%%%%%%%%%%%%%%%%%%%%%%%%%%%%%%%%%%%%%%
\section{Tutorial style}
\index{Style, tutorial}
\index{Tutorial, style}
%%%%%%%%%%%%%%%%%%%%%%%%%%%%%%%%%%%%%%%%%%%%%%%%%%%%%%%%%%%%%%%%%%%%%%%%%%%%%%%%

\paragraph{Readable for beginners}
\index{Readability, for beginners}

This tutorial is aimed at the beginner programmer.
 
\paragraph{High verbosity}

This tutorial is intended to be as verbose, 
such that a beginner should
be able to follow every step, 
from reading the tutorial from beginning
to end chronologically.
Especially in the earlier chapters, the rationale behind the code presented
is given, including references to the literature.

\paragraph{Repetitiveness}

This tutorial is intended to be as repetitive, such that a beginner can
spot the patterns in the code snippets their increasing complexity.
Extending code from this tutorial should be as easy as extending the patterns.

%%%%%%%%%%%%%%%%%%%%%%%%%%%%%%%%%%%%%%%%%%%%%%%%%%%%%%%%%%%%%%%%%%%%%%%%%%%%%%%%
\section{Coding style}
\index{Coding style}
\index{Style, coding}
%%%%%%%%%%%%%%%%%%%%%%%%%%%%%%%%%%%%%%%%%%%%%%%%%%%%%%%%%%%%%%%%%%%%%%%%%%%%%%%%

All code is tested.

\paragraph{Coding standard}
\index{Coding standard}

I use the coding style from the  \verb{nf-core} linter.

\paragraph{No comments in code}
\index{Comments}
\index{Readability, no comments in code}

It is important to add comments to code.
In this tutorial, however, I have chosen not to put comments in code, 
as I already describe the function in the tutorial its text.
This way, it prevents me from saying the same things twice.

\paragraph{Long function names}
\index{Long function names}

I enjoy to show concepts by putting those in (long-named) functions.
These functions sometimes border the trivial, by, for example, only calling
a single function.
On the other hand, these functions have more English-sounding names, resulting
in demonstration code that is readable.
Additionally, they explicitly mention their return type (in a simpler way),
which may be considered informative.

\paragraph{Re-use of functions}

The functions I develop in this tutorial are re-used from that moment on.
This improves to readability of the code and decreases the number of lines.

\paragraph{Tested to work}
\index{Tested examples}

All code in this tutorial is tested.
GitHub Actions calls these tests after each push to the repository.

\paragraph{Availability}
\index{Availability, code}
\index{Availability, text}
\index{Download}

The code, as well as this tutorial, can be downloaded from the GitHub at
\url{www.github.com/richelbilderbeek/nextflow_cookbook_1}.

%%%%%%%%%%%%%%%%%%%%%%%%%%%%%%%%%%%%%%%%%%%%%%%%%%%%%%%%%%%%%%%%%%%%%%%%%%%%%%%%
\section{License}
\index{License}
%%%%%%%%%%%%%%%%%%%%%%%%%%%%%%%%%%%%%%%%%%%%%%%%%%%%%%%%%%%%%%%%%%%%%%%%%%%%%%%%

This tutorial is licensed under Creative Commons license 4.0.

\begin{figure}[!htbp]
  \includegraphics[]{CC-BY-SA_icon.png}
  \caption{
    Creative Commons license 4.0
  }
  \label{fig:license}
\end{figure}

\section{Feedback}

This tutorial is not intended to be perfect yet.
For that, I need help and feedback from the community.
All referenced feedback is welcome, as well as any constructive feedback.

I have tried hard to strictly follow the style as described above.
If you find I deviated from these decisions somewhere, I would be grateful
if you'd let know.
Next to this, there are some sections that need to be coded or have its
code improved.

%%%%%%%%%%%%%%%%%%%%%%%%%%%%%%%%%%%%%%%%%%%%%%%%%%%%%%%%%%%%%%%%%%%%%%%%%%%%%%%%
\section{Acknowledgements}
\index{Acknowledgements}
%%%%%%%%%%%%%%%%%%%%%%%%%%%%%%%%%%%%%%%%%%%%%%%%%%%%%%%%%%%%%%%%%%%%%%%%%%%%%%%%

These are users that improved this tutorial and/or the code behind this
tutorial, in chronological order:

\begin{itemize}
  \item none yet
\end{itemize}

%%%%%%%%%%%%%%%%%%%%%%%%%%%%%%%%%%%%%%%%%%%%%%%%%%%%%%%%%%%%%%%%%%%%%%%%%%%%%%%%
\section{Outline}
%%%%%%%%%%%%%%%%%%%%%%%%%%%%%%%%%%%%%%%%%%%%%%%%%%%%%%%%%%%%%%%%%%%%%%%%%%%%%%%%

The chapters of this tutorial are also like a well-connected graph.
To allow for quicker learners to skim chapters, or for beginners looking
to find the patterns.
